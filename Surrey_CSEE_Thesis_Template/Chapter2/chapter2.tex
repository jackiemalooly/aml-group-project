\chapter{Demos}
\section{Math}
\subsection{Inline Math}

Inline math is used when you want to include equations within the flow of your text. For example, the mean squared error loss function is calculated as:
\begin{math}
\sum_{i=1}^{D}(x_i-y_i)^2
\end{math}
This equation is embedded inline with the text.

\subsection{Display Math}
Display math is used when you want to present a formula on its own line. Here are some examples:
\begin{equation}
L_{\delta} =
    \left\{
    \begin{matrix}
        \frac{1}{2}(y - \hat{y})^{2} & \text{if } \left | (y - \hat{y}) \right | < \delta \\
        \delta ((y - \hat{y}) - \frac{1}{2} \delta) & \text{otherwise}
    \end{matrix}
    \right.
\end{equation}
This presents the Huber loss function on a separate line.

A few more examples:
\begin{equation}
    MI(U,V) = \sum_{i=1}^{|U|} \sum_{j=1}^{|V|} \frac{|U_i \cap V_j|}{N}
\log\frac{N|U_i \cap V_j|}{|U_i||V_j|}
\end{equation}

\begin{equation}\label{equ:someequation}
    \text{Sensitivity} = \text{Recall} = \frac{TP}{TP + FN} \\
    \text{Specificity} = \frac{TN}{FP + TN}
\end{equation}
To reference equations in your text, use the label parameter as shown above and refer to it using \ref{equ:someequation}.

If you want to exclude an equation from numbering, use the `*` parameter:
\begin{equation*}
    1 + 1 = 2
\end{equation*}

\section{Tables}
You can create tables using the following structure. Adjust the columns and rows as needed in the visual editor of Overleaf.

\begin{table}[ht]
    \centering
    \begin{tabular}{|c|>{\centering\arraybackslash}p{0.7\linewidth}|}
        \hline 
        Parameter & Position \\ \hline 
        h & Place the float here, i.e., approximately at the same point it occurs in the source text (however, not exactly at the spot) \\ 
        t & Position at the top of the page. \\ 
        b & Position at the bottom of the page. \\ 
        p & Put on a special page for floats only. \\ 
        ! & Override internal parameters \LaTeX{} uses for determining "good" float positions. \\ 
        H & Place the float at precisely the location in the \LaTeX{} code. Requires the float package. This is somewhat equivalent to h!. \\ \hline
    \end{tabular}
    \caption{Parameters for float positioning. These apply to images and tables as well.}
    \label{tab:sometable}
\end{table}

To reference a table, use \ref{tab:sometable}. It is good practice to prefix labels with their type (e.g., `tab` for tables, `fig` for figures).

\section{Images}
\begin{figure}[!ht]
    \centering
    \includegraphics[width=.49\linewidth]{Figures/Surrey Logo.png}
    \includegraphics[width=.49\linewidth]{Figures/Surrey Logo.png}
    \caption{A demonstration of tiling two images together.}
    \label{fig:somefigure}
\end{figure}

This demonstrates how to tile images together as one figure. You can adjust the number of images and their widths as needed. To reference this figure, use \ref{fig:somefigure}.

\section{Lists}
\subsection{Unordered Lists}
\begin{itemize}
    \item This
    \item is
    \item how
    \item you
    \item create
    \item an
    \item unordered
    \item list
\end{itemize}

\subsection{Ordered Lists}
\begin{enumerate}
    \item This
    \item is
    \item an
    \item ordered
    \item list
\end{enumerate}

\section{Code}
\begin{verbatim}
somepythonlist = []
for i in range(0, 1000):
    print(i)
    somepythonlist.append(i)

\LaTeX
\cite{turing1950mind}
\end{verbatim}

This demonstrates how to include code segments in \LaTeX{}. All commands are ignored in `verbatim` mode, displaying exactly what you write.

\section{References and Others}
\subsection{References}
Citing references adds them to the bibliography section. Use `\cite{turing1950mind}` to cite a reference. Ensure your `bibliography.bib` file includes the proper BibTeX entries. Papers from Google Scholar or publishers often provide a BibTeX citation.

Other examples of references:
\cite{hopper1952education}, \cite{hamiltonsoftware}

\subsection{Acronyms}
To use acronyms, link them in the abbreviations file:
\acrfull{ieee} for the full name. Use `\acrlong{ieee}` for just the name, or `\acrshort{acm}` for the short form. Ensure acronyms are defined in `abbreviations.tex`.

\section{Etc. and Final Words}\label{finalwords}
\href{https://www.overleaf.com/learn/latex/Learn_LaTeX_in_30_minutes}{Here is a link for learning more about \LaTeX{}}. It's a great reference tool with many examples. 

Writing a thesis is challenging, but remember to take breaks, de-stress, and take care of yourself. Good luck!

Thanks for reading,\\
Creator: Aaron\\
Edit \& Consistency: Alireza (\LaTeX\ 2024)
